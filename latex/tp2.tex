\documentclass[8pt,a4paper]{article}
\usepackage{clrscode}
\usepackage[conEntregas]{tp2}
\usepackage[spanish]{babel} % para que comandos como \today den el resultado en castellano
\usepackage{a4wide} % márgenes un poco más anchos que lo usual
\usepackage[T1]{fontenc}
\usepackage{textcomp}
\usepackage{graphicx}
\usepackage{enumitem}
\newcommand{\subscript}[2]{$#1 _ #2$}
\usepackage[utf8]{inputenc} 
\usepackage{pdfpages}
\usepackage{amsmath}
\usepackage{vmargin}
\setpapersize{A4}


\begin{document}
\titulo{Trabajo Práctico 2}
\subtitulo{Clustering}

\fecha{\today}

\materia{Algoritmos y Estructura de Datos III}

\integrante{Buceta, Diego}{001/17}{diegobuceta35@gmail.com}
\integrante{Springhart, Gonzalo}{308/17}{glspringhart@gmail.com}
\integrante{Tasat, Dylan}{/17}{@gmail.com}
\integrante{Gabriel, Jimenez}{/17}{@gmail.com}
% Pongan cuantos integrantes quieran

\maketitle

\newpage
%\includepdf[pages={1,2}]{tp1.pdf}

\section{Introducción al problema}
\section{Justificación teórica}
\section{Algoritmos presentados}
\subsection{Complejidad}
\section{Experimentación}


\end{document}
